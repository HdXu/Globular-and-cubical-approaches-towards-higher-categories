\documentclass{article}
\usepackage[utf8]{inputenc}
\usepackage{ifxetex,ifluatex}
\newif\ifxetexorluatex
\ifxetex
  \xetexorluatextrue
\else
  \ifluatex
    \xetexorluatextrue
  \else
    \xetexorluatexfalse
  \fi
\fi

\ifxetexorluatex
  \usepackage{fontspec}
\else
  \usepackage[T1]{fontenc}
  \usepackage[utf8]{inputenc}
  \usepackage{lmodern}
\fi

\usepackage{hyperref}
\usepackage{tikz-cd}
\usetikzlibrary{matrix,arrows}
\usepackage{amsmath}
\usepackage{amssymb}
\usepackage{mathtools}
\usepackage{amsthm}
\usepackage{tensor}
\usepackage{bbm}
\usepackage{enumitem}% http://ctan.org/pkg/enumitem

\theoremstyle{definition}
\newtheorem{mydef}{Definition}[section]
\newtheorem{prop}{Proposition}[section]
\newtheorem{thm}[prop]{Theorem}
\newtheorem*{eg}{Example}
\newtheorem*{cor}{Corollary}
%\newenvironment{proof}{\paragraph{Proof:}}{\hfill$\square$}

\theoremstyle{remark}
\newtheorem*{remark}{Remark}

\title{Globular and cubical approaches towards higher categories}
\author{Hao Xu}
\date{August 2019}

\begin{document}

\maketitle

\section{Introduction}

Category theory studies various mathematical structures and structure-preserving transformations. A general principle in studying a particular type of mathematical objects is that instead of asking whether two objects are \textbf{equal}, in a set-theoretical sense, we should really ask whether two objects are \textbf{equivalent}, i.e. there exists a pair of structure preserving transformation between them, and the compositions \textbf{equal} identity transformations. In other word, we replace equivalence between two objects from a \textbf{judgment} to a \textbf{construction} via information in arrows. It is not hard to see how to generalize this principle further - instead of comparing two arrows as elements in a set, we should consider existences of ’arrows between arrows’, i.e. $2$-morphisms, such that two arrows are equivalent if and only if there exists a pair of $2$-arrows such that their compositions give identity on two $1$-arrows respectively. Of course, there are some details we need to clarify in the above discussion: how to send a $n$-arrow to its identity (which should \textbf{in apriori} be a $(n+1)$-arrow) and how to composite two $n$-arrows to get a new $n$-arrows. Higher category theory tries to provide an answer to above questions in a systematic way, and so far we have obtained several models of higher categories:

\section{Complete $n$-fold Segal spaces}

Before we go to $n$-fold complete Segal spaces, we first look at the most basic case: ($1$-fold) complete Segal spaces.

\begin{mydef}

The simplex category, $\Delta$, is the category of \textbf{non-empty} finite ordinals and order-preserving maps between them. Let $[k]$ denote the ordinal with $k+1$ elements. This category is generated by the following two families of morphisms:

\begin{enumerate}
    \item $\delta_i: [n] \to [n+1]$ where $0 \leq i \leq n+1$ such that $$ \delta_i(k):= \begin{cases} k, & 0 \leq k < i \\ k+1, & i \leq k \leq n \end{cases} $$
    \item $\sigma_i: [n+1] \to [n]$ where $0 \leq i \leq n$ such that $$ \sigma_i(k):= \begin{cases} k, & 0 \leq k \leq i \\ k-1, & i < k \leq n+1 \end{cases} $$
\end{enumerate}

\end{mydef}

\begin{mydef}

The category of contravariant functors from the simplex category $\Delta$ to the category of (small) sets $\mathsf{Set}$ is called the \textbf{category of simplicial sets} (with simplicial maps between them), denoted as $\mathsf{sSet}$. We still denote the Yoneda embeddings of ordinal $[k]$ in $\mathsf{sSet}$ as $\Delta^k$, called standard $k$-simplex. For any simplicial set $X: \Delta^{op} \to \mathsf{Set}$, we denote images of $[k]$ as $X_k$, images of $\delta_i: [n] \to [n+1]$ as $\partial_i: X_{n+1} \to X_n$ (so called \textbf{face maps}), images of $\sigma_i: [n+1] \to [n]$ as $s_i: X_n \to X_{n+1}$ (so called \textbf{degeneracy maps}). Then one can prove the following identities on simplicial sets, called \textbf{simplicial identities}.

\begin{enumerate}
    \item for $i \leq j$, $$ \partial_i \partial_{j+1} = \partial_j \partial_i $$
    \item for $i < j$, $$ s_i s_j = s_{j+1} s_i $$
    \item $$ \partial_i s_j = \begin{cases} s_{j-1} \partial_i, & i<j \\ \mathrm{Id}, & i=j,j+1 \\ s_j \partial_{i-1}, & i>j+1 \end{cases}$$
\end{enumerate}

\end{mydef}

Small limits and colimits all exist in $\mathsf{sSet}$, and it is locally finitely presentable. In particular, \textbf{product} of any two simplicial sets $X,Y$ can be defined to be $$(X \times Y)_n := X_n \times Y_n$$ with face and degeneracy maps: $$\partial_{X \times Y,i}:=\partial_{X,i} \times \partial_{Y,i}$$   $$s_{X \times Y,i}:=s_{X,i} \times s_{Y,i}$$ i.e. product of simplicial sets is given as product of functors.

$\mathsf{sSet}$ with product defined above forms a symmetric monoidal category, whose monoidal unit is just the singleton (i.e. standard $0$-simplex $\Delta^0$). One can show that this monoidal product is closed, thus there exists a simplicial \textbf{internal hom} $ [\cdot,\cdot]: \mathsf{sSet}^{op} \times \mathsf{sSet} \to \mathsf{sSet}$; for any simplicial set $X,Y$, we can define $$ [X,Y]_k := \mathbf{Hom}_{\mathsf{sSet}}(\Delta
^k \times X,Y)$$

Historically, simplicial sets arise as an alternative to some more classical combinatorial models of spaces via triangulation (such as simplicial complexes). This deep connection with topological spaces can be made precise by the following constructions of \textbf{nerve and realization}.

\begin{mydef}

\textbf{Geometric standard $n$-simplex} is the following subset of $\mathbb{R}^{n+1}$: $$ \triangle^k:= \{(x_0,\dots,x_n) \in \mathbb{R}^{n+1}: x_0 + \cdots + x_n = 1, 0 \leq x_i \leq 1 \quad (0 \leq i \leq n)\} $$ This space is embedded in $\mathbb{R}^n$ as subspace, since it lies on the hyperplane defined by the equation $x_0 + \cdots + x_n = 1$, which is homeomorphic to $\mathbb{R}^n$. As for combinatorial standard simplexes, we can define continuous face and degeneracy maps between geometric standard simplexes as following: for any non-negative integer $n$, $$ D_i: \triangle^n \to \triangle^{n+1} \qquad 0 \leq i \leq n+1 $$    $$ (x_0,\dots,x_n) \mapsto (x_0,\dots,x_{i},0,x_{i+1},\dots,x_n) $$    $$ S_i: \triangle^{n+1} \to \triangle^n \qquad 0 \leq i \leq n $$    $$ (x_0,\dots,x_{n+1}) \mapsto (x_0,\dots,\hat{x_{i}},\dots,x_{n+1}) $$ where $\hat{x_{i}}$ means delete the term $x_i$ from the list.

One can verify that the subcategory in $\mathsf{Top}$ (the category of topological spaces and continuous maps between them) spanned by geometric standard simplexes with face and degeneracy maps between them is equivalent to the simplex category $\Delta$ (where the name comes from). In other word, we have an embedding of categories $$ \iota: \Delta \to \mathsf{Top} $$ which sends $[n]$ to $\triangle^n$, face and degeneracy maps to corresponding ones.

\end{mydef}

\begin{mydef}

The left Kan extension of the above embedding $\iota$ along the Yoneda embedding functor $\Upsilon: \Delta \to \mathsf{sSet}$ gives the \textbf{geometric realization} functor $|\cdot|: \mathsf{sSet} \to \mathsf{Top}$; the left Kan extension of the Yoneda embedding functor $\Upsilon: \Delta \to \mathsf{sSet}$ along $\iota$ gives the \textbf{singular complex} functor $\mathrm{Sing}: \mathsf{Top} \to \mathsf{sSet}$.

$$\begin{tikzcd} \Delta \arrow[d, "\Upsilon"', hook] \arrow[r, "\iota", hook] & \mathsf{Top} \arrow[ld, "\mathrm{Sing}", dotted, shift left] \\
\mathsf{sSet} \arrow[ru, "|\cdot|", dashed, shift left]               &                                         
\end{tikzcd}$$

Unwind these definitions, we can show that for any individual simplicial set (resp. topological space): $$ |\cdot|:X \mapsto \int^{[k]:\Delta}{\triangle^k \times \mathrm{discrete}(X_k) }$$    $$ \mathrm{Sing}: Y \mapsto ([k] \mapsto \mathbf{Hom}_{\mathsf{Top}}(\triangle^k,Y))$$

\end{mydef}

\begin{prop}

Singular complex functor is right adjoint to geometric realization functor.

\end{prop}

\begin{proof}
For any simplicial set $X$ and topological space $Y$, we have 
\begin{eqnarray*} \mathbf{Hom}_{\mathsf{Top}}(|X|,Y) &\cong& \mathbf{Hom}_{\mathsf{Top}}\left(\int^{[k]:\Delta}{\triangle^k \times \mathrm{discrete}(X_k) },Y\right) \\ &\cong& \int_{[k]:\Delta}{\mathbf{Hom}_{\mathsf{Top}}( \triangle^k \times \mathrm{discrete}(X_k),Y) } \\ &\cong&  \int_{[k]:\Delta}{\mathbf{Hom}_{\mathsf{Set}}(X_k, \mathbf{Hom}_{\mathsf{Top}}(\triangle^k,Y)}) \\ &\cong& \int_{[k]:\Delta}{\mathbf{Hom}_{\mathsf{Set}}(X_k,\mathrm{Sing}(Y)_k)}\\ \\ &\cong& \mathbf{Hom}_{\mathsf{sSet}}(X,\mathrm{Sing}(Y)) \end{eqnarray*} where the last isomorphism comes from the enriched end construction of enriched natural transformations. The naturality of above sequence of isomorphisms can be verified easily. Therefore, we have the adjoint pair $(|\cdot| \dashv \mathrm{Sing}$).

\end{proof}

In algebraic topology, fundamental group of topological spaces and its higher dimensional analogues ($n$-homotopy groups) play an important role in classifying spaces into different (homotopy) types. The counit of the above adjunction assigns a continuous mapping to each topological space $X$, $\epsilon_X: |\mathrm{Sing}(X)| \to X $, which in turn induces isomorphisms of homotopy groups at all levels, i.e. for any $n \in \mathbb{N}$, $$ (\epsilon_X)_*: \pi_n(|\mathrm{Sing}(X)|) \xrightarrow{\sim} \pi_n(X)$$ In other word, geometric realization of singular complex of a space is weakly homotopy equivalent to itself. Hence, viewed as a combinatorial counterpart of $\mathsf{Top}$, simplicial sets could have their homotopy groups defined as homotopy groups of their geometric realizations. As for topological spaces, we say two simplicial sets $X,Y$ are \textbf{weak homotopy equivalent} (denoted as $X \simeq Y$) if there is a continuous map between their geometric realizations $f: |X| \to |Y|$ such that for any $n \in \mathbb{N}$, $f_*:\pi_n(|X|) \xrightarrow{\sim} \pi_n(|Y|)$ is an isomorphism. The fact is weak homotopy equivalences of simplicial sets is indeed an equivalence relation (one way of showing this is to put a model category structure on $\mathsf{sSet}$, such as Quillen’s model category structure). Localization of $\mathsf{sSet}$ with respect to weak homotopy equivalences gives a category equivalent to the classical homotopy category of topological spaces; for details readers should consult Quillen adjunction and Quillen equivalence of model categories.

Simplicial sets are also related with ordinary (small) categories. Conducting a similar procedure, we define the $\textbf{nerve}$ of (small) categories to be the left Kan extension of the Yoneda embedding $\Upsilon:\Delta \to \mathsf{sSet}$, along the embedding $\iota: \Delta \to \mathsf{Cat}$ (where we view each non-empty finite ordinal as linearly ordered set, hence also a small category) $$ \begin{tikzcd} \Delta \arrow[d, "\Upsilon"', hook] \arrow[r, "\iota", hook] & \mathsf{Cat} \arrow[ld, "\mathbf{N}", dotted] \\ \mathsf{sSet} \end{tikzcd} $$

In a concrete construction, the nerve of a (small) category $C$ is a simplicial set, whose set of $k$-simplexes consists of composable $k$-tuples of morphisms, i.e. $\mathbf{N}(C)_0$ is the set of objects in $C$, $\mathbf{N}(C)_1$ is the set of morphisms in $C$, then for any integer $k \geq 2$, the upper left square in the following diagam is a pullback diagram: $$ \begin{tikzcd} \mathbf{N}(C)_k \arrow[r] \arrow[d] & \mathbf{N}(C)_{k-1} \arrow[d,"\mathrm{dom}"] \arrow[r,"\mathrm{cod}"] & \mathbf{N}(C)_0 \\ \mathbf{N}(C)_1 \arrow[d,"\mathrm{dom}"] \arrow[r,"\mathrm{cod}"] & \mathbf{N}(C)_0 \\ \mathbf{N}(C)_0 \end{tikzcd}$$ $s_0:\mathbf{N}(C)_0 \to \mathbf{N}(C)_1$ is given by sending each object to its identity morphism; $\partial_1: \mathbf{N}(C)_1 \to \mathbf{N}(C)_0 $ and $\partial_0:\mathbf{N}(C)_1 \to \mathbf{N}(C)_0 $ are given by sending each morphism to its domain and codomain, respectively. $\partial_1: \mathbf{N}(C)_2 \to \mathbf{N}(C)_1 $ is given by sending each pair of composable morphisms $(f,g)$ with $\mathrm{cod}(f)=\mathrm{dom}(g)$ to their composition $g \circ f$. Then this information completely determine the nerve $\mathbf{N}(C)$.

\begin{mydef}

\textbf{Geometric horn} $\Lambda^n_k$ (where $n$ is a non-negative integer, $0 \leq k \leq n$) is defined as following: $$ \Lambda^n_k:=\{(x_0,\dots,x_n) \in \triangle^n: \text{for at least one of }i, x_i=0, (i \ne k, 0 \leq i \leq n)\}$$ The obvious embedding map $$ \Lambda^n_k \hookrightarrow \triangle^n $$ induces an inclusion of simplicial sets: $$ \mathrm{Sing}(\Lambda^n_k) \hookrightarrow \mathrm{Sing}(\triangle^n)$$ It is immediate that $\Delta^n \simeq \mathrm{Sing}(\triangle^n)$. We define a \textbf{(combinatorial) horn} $\Lambda^n_k$ as any simplicial set that is weakly homotopy equivalent to the singular complex of the corresponding geometric horn, with an inclusion $$ \Lambda^n_k \hookrightarrow \Delta^n $$ homotopic to the above inclusion. Passing to the level of homotopy category, they all represent the same homotopy class of maps in the mapping space, $$\mathbf{Hom}_{\mathrm{ho}(\mathsf{sSet})}([\mathrm{Sing}(\Lambda^n_k)],[\mathrm{Sing}(\triangle^n)]) \cong \mathbf{Hom}_{\mathrm{ho}(\mathsf{Top})}([\Lambda^n_k],[\triangle^n])$$

In practice, we can use a simple combinatorial definition of horn $\Lambda^n_k$ from standard $n$-simplex, by deleting the unique non-degenerate $n$-simplex and the non-degenerate $(n-1)$-simplex corresponding to the $k$-th face in $\Delta^n$.

\end{mydef}

\begin{mydef}

A simplicial set $X$ is called a \textbf{Kan complex} if it has right lifting property against all horn inclusions, i.e. for any $n \in \mathbb{N}$, $ 0 \leq i \leq n$, given a simplicial map $f: \Lambda^n_k \to \Delta^n$, there eixsts an extension $L: \Delta^n \to X$  such that the following diagram commutes:
$$ \begin{tikzcd} \Lambda^n_i \arrow[d, hook] \arrow[r,"f"] & X \\ \Delta^n \arrow[ru, "\exists L"', dashed] \end{tikzcd} $$

The full subcategory spanned by Kan complexes in $\mathsf{sSet}$ is denoted as $\mathsf{Kan}$.

\end{mydef}

\begin{prop}
$\mathsf{Kan}$ is closed under product of $\mathsf{sSet}$.
\end{prop}

\begin{proof}
Suppose $X$ and $Y$ are two Kan complexes. Then for any horn inclusion $\Lambda^n_i \hookrightarrow \Delta^n$ (where $n \in \mathbb{N}, 0 \leq i \leq n$) and a simplicial map $f:\Lambda^n_i \to X \times Y$, we have extensions of $\pi_X \circ f,\pi_Y \circ f$ along the horn inclusion, denoted as $L_X,L_Y$ respectively. Then by the universal property of product, there exists a unique extension $L: \Delta \to X \times Y$ such that $\pi_X \circ \Delta = L_X$ and $\pi_Y \circ \Delta = L_Y$, i.e. $X \times Y$ is a Kan complex. $$ \begin{tikzcd} & \Delta^n \arrow[ld, "\exists L_X"', dashed] \arrow[rd, "\exists L_Y", dashed] &   \\ X & \Lambda^n_i \arrow[d, "f"] \arrow[l] \arrow[r] \arrow[u, hook] & Y \\ & X \times Y \arrow[lu, "\pi_X"] \arrow[ru, "\pi_Y"'] & \end{tikzcd} $$

\end{proof}

\begin{remark}
Kan complexes are fibrant objects in the category of simplicial sets with Quillen's model category structure, i.e. the cofibrantly generated model category structure on $\mathsf{sSet}$ with generating cofibrations $$ \partial \Delta^n \hookrightarrow \Delta^n $$ and generating trivial cofibrations $$ \Lambda^n_i \hookrightarrow \Delta^n$$ where $n \in \mathbb{N}$ and $ 0 \leq i \leq n$. Then the above proposition results follow directly from the properties of fibrant subcategory.
\end{remark}

\begin{prop}
The nerve of any groupoid is a Kan complex.
\end{prop}

\begin{proof}

\end{proof}

\begin{prop}
$\mathsf{Kan}$ is closed under internal hom of $\mathsf{sSet}$.(...)
\end{prop}

\begin{proof}

For any simplicial map $g: \Lambda^n_i \to [X,Y]$, using adjunction and symmetry of monoidal product, we have $\hat{g}: X \to [\Lambda^n_i,Y]$. Since $Y$ is a Kan complex, fix a choice of extension for each extension problem: $\Delta^n \hookleftarrow \Lambda^n_i \xrightarrow{f} Y$, so there exists a map between sets: $$\mathcal{L}:\mathbf{Hom}_{\mathsf{sSet}}(\Lambda^n_i,Y) \to \mathbf{Hom}_{\mathsf{sSet}}(\Delta^n,Y)$$    $$ (\Lambda^n_i \xrightarrow{f} Y) \mapsto \left( \begin{tikzcd} \Lambda^n_i \arrow[d, hook] \arrow[r,"f"] & Y \\ \Delta^n \arrow[ru, "\mathcal{L}(f)"', dashed] \end{tikzcd} \right) $$ Meanwhile, the horn inclusion induces $\Lambda^n_i \times X \hookrightarrow \Delta^n \times X$. Notice that $$ (\Lambda^n_i \times X)_k \cong \coprod_{a:X_k}{(\Lambda^n_i)_k}$$    $$ (\Delta^n \times X)_k \cong \coprod_{a:X_k}{(\Delta^n)_k}$$
$$ \begin{tikzcd} \Lambda^n_i \arrow[d, hook]  & \Lambda^n_i \times X \arrow[d, hook] \arrow[r, "\hat{g}"] \arrow[l] & Y \\ \Delta^n \arrow[rru, dotted] & \Delta^n \times X \arrow[ru, "\exists \hat{L}"', dashed] \arrow[l]  & \end{tikzcd} $$

\end{proof}


\begin{prop}[\textbf{Whitehead Theorem}]
Weak homotopy equivalences and homotpopy equivalences between Kan complexes coincide.
\end{prop}

\begin{proof}

\end{proof}

\begin{mydef}
A \textbf{Segal space} is a simplicial object in $\mathsf{Kan}$, i.e. $X: \Delta^{op} \to \mathsf{Kan}$ such that it satisfies the Segal condition: for any integer $s,t \geq 1$, the pushout diagram in simplex category $\Delta$ $$ \begin{tikzcd}   {[s+t]}  & {[s]} \arrow[l] \\ {[t]} \arrow[u] & {[0]} \arrow[l, "\delta_{t} \cdots \delta_1"] \arrow[u, "\delta_0 \cdots \delta_0"']  \end{tikzcd} $$ induces a $\textbf{homotopy pullback diagram}$ in $\mathsf{Kan}$ $$ \begin{tikzcd} X_{s+t} \arrow[d] \arrow[r] & X_s \arrow[d] \\ X_t \arrow[r] & X_0 \end{tikzcd} $$ i.e. the canonical map (to the homotopy pullback), $ X_{s+t} \to X_s \times_{X_0} X_t $, is a weak homotopy equivalence. We denote the category of Segal spaces (with natutal transformations between them) as $\mathsf{SS}_1$.

Let $E$ be the connected groupoid with two objects a pair of non-trivial isomorphisms between them, and $\mathrm{N}E$ be its nerve. There is a natural embedding of categories $\mathsf{Set} \hookrightarrow \mathsf{Kan}$, sending each set to the discrete groupoid generated by its elements; so this induces an embedding of simplicial sets into simplicial spaces $\mathbf{Fun}(\Delta^{op},\mathsf{Set}) \hookrightarrow \mathbf{Fun}(\Delta^{op},\mathsf{Kan})$. The image of the above embedding lies in the subcategory of Segal spaces. Hence, there is a simplicial enriched hom funcor in the category of Segal spaces $$ [\cdot,\cdot]:\mathsf{SS}_1^{op} \times \mathsf{SS}_1 \to \mathsf{sSet} $$ such that for any Segal spaces $X,Y$, the enriched hom is defined levelwise to be $$[X,Y]_k:=\mathbf{Hom}_{\mathsf{SS}_1}(\Delta^k \times X,Y)$$

Suppose $X: \Delta^{op} \to \mathsf{Kan}$ is a Segal space, then it is said to be \textbf{complete} if it satisfies the following condition. The unique map of Segal spaces $\mathrm{N}E \to *$ induces a map of simplicial sets $$ [*,X] \to [\mathrm{N}E,X] $$ which is a weak homotopy equivalence. Unwinding this condition, we find $X_0 \simeq [*,X]$ and $[\mathrm{N}E,X]$ can be viewed as \textit{formal equivalences} in $X$ (indeed, when $X$ is the nerve of a category, delooping of $[\mathrm{N}E,X]$ is isomorphic to the nerve of the maximal subgroupoid in $X$, up to retract). In other word, we require any formal equivalence in $X$ to come from the $0$-th level, i.e. $X$ has no non-trivial equivalences (somtimes we call this property as being \textbf{gaunt}). The full subcategory spanned by complete Segal space is denoted as $\mathsf{CSS}_1$.

\end{mydef}

\begin{remark}
Complete Segal spaces are one of the models for the theory of $(\infty,1)$-categories. Other models for $(\infty,1)$-categories include quasi-categories, complete Segal $\Theta$-spaces, topological categories, simplicial categories, Segal categories, marked simplicial sets, relative categories, etc. There are also references on the proofs that these models are all Quillen equivalent to each other. When generalized to higher infinity categories, we also have models such as $n$-fold complete Segal spaces (which we shall emphasize throughout this note), complete Segal $\Theta_n$-spaces, relative $n$-categories, etc.
\end{remark}

\begin{mydef}
A \textbf{$n$-fold complete Segal space} is a $n$-simplicial space, i.e. $X: (\Delta^n)^{op} \to \mathsf{Kan}$ which satisfies the following conditions:

\begin{enumerate}
    \item  \textbf{[Globe condition]} For $0 \leq k \leq n-1$, fix first $k$ coordinates of $X$  ($a_1,\dots,a_k \in \mathbb{N}$), then $X_{a_1,\cdots,a_k,0,-,\cdots,-}: (\Delta^{n-k-1})^{op} \to \mathsf{Kan}$ is essentially constant, i.e. it is weakly homotopy equivalent to the image of some Kan complex under the constant embedding functor $$ c_{n-k-1}: \mathsf{Kan} \hookrightarrow \mathbf{Fun}((\Delta^{n-k-1})^{op},\mathsf{Kan})$$
    \item  \textbf{[Segal condition]} $X$ satisfies Segal condition in each of its coordinates, i.e. (suppose the following subscripts appear in the same coordinate) for any integers $s,t \geq 1$, $X_{\cdots,s+t,\cdots} \to X_{\cdots,s,\cdots} \times_{X_{\cdots,0,\cdots}} X_{\cdots,t,\cdots}$ is a weak homotopy equivalence.
    \item  \textbf{[Completeness condition]} $X_{1,-,\cdots,-}: (\Delta^{n-1})^{op} \to \mathsf{Kan}$ is a a $(n-1)$-fold complete Segal space; also, $X_{-,0,\cdots,0}: \Delta^{op} \to \mathsf{Kan}$ is complete Segal space.
\end{enumerate}

\end{mydef}

\begin{remark}
The above definition of $n$-fold complete Segal spaces is recursive, hence there are some subtleties we need to clarify. Recall that the simplex category $\Delta$ is a \textbf{Reedy category}, and $\mathsf{Kan}$ has a \textbf{combinatorial model category structure} inherited from that of simplicial sets $\mathsf{sSet}$. So there are various choices of model category structures which we can put on the  functor category $\mathbf{Fun}((\Delta^k)^{op},\mathsf{Kan})$ ($1 \leq k \leq n$): injective, projective or Reedy model category structures.
\end{remark}

\begin{remark}
Informally, we can understand the above conditions like this. The globe condition implies that $\mathsf{CSS}_n$ contains models of all \textbf{$k$-globes} (for details readers may go to Appendix A), for $0 \leq k \leq n$: $$\Upsilon(\underbrace{[1],\dots,[1]}_{k \; \text{copies}},[0],\dots,[0]) \Longleftrightarrow k \; \text{globe}$$ where $ \Upsilon:\Delta^n \hookrightarrow \mathbf{Fun}((\Delta^n)^{op},\mathsf{Kan}) \to \mathsf{CSS}_n $ is the (Kan-complex-valued) Yoneda embedding followed by localization to complete Segal spaces. Since $ \mathbf{Fun}( (\Delta^n)^{op},\mathsf{Kan}) $ is generated (up to weak equivalence) by $k$-globes under homotopy colimits, the globe condition literaturely means we are modeling $(\infty,n)$-categories under a globular picture. For example, if $X$ is a $2$-fold complete Segal space, then $X_{1,1}$ is the space of all $2$-morphisms in $X$, presented as $2$-globes:

$$\begin{tikzcd}[column sep=huge,row sep=huge] 
A 
    \arrow[d, "f"'] 
    \arrow[r,-,"\sim"] 
& {A'} 
    \arrow[d,"g"] \\ 
B
    \arrow[r,-,"\sim"] 
    \arrow[shorten <=10pt,shorten >=10pt,ru,Rightarrow] 
& {B'} 
\end{tikzcd} \quad = \quad
\begin{tikzcd}[column sep=huge,row sep=huge]
{A} 
    \arrow[bend right]{r}[name=D,label=below:$g$]{}
    \arrow[bend left]{r}[name=U,label=above:$f$]{}
& {B}
    \arrow[shorten <=3pt,shorten >=3pt,Rightarrow,to path={(U) -- node[label=right:$\alpha$] {} (D)}]{}
\end{tikzcd} , $$

$$ \begin{tikzcd}[column sep=huge,row sep=huge] 
C 
    \arrow[d, "h"'] 
    \arrow[r,-,"\sim"] 
& {C'} 
    \arrow[d,"k"] \\ 
D 
    \arrow[r,-,"\sim"] 
    \arrow[shorten <=10pt,shorten >=10pt,ru,Rightarrow] 
& {D'} 
\end{tikzcd} \quad = \quad
\begin{tikzcd}[column sep=huge,row sep=huge]
{C} 
    \arrow[bend right]{r}[name=D,label=below:$k$]{}
    \arrow[bend left]{r}[name=U,label=above:$h$]{}
& {D}
    \arrow[shorten <=3pt,shorten >=3pt,Rightarrow,to path={(U) -- node[label=right:$\beta$] {} (D)}]{}
\end{tikzcd} , $$

$$ \cdots $$

$X_{2,3}$ is the space of all $2 \times 3$ grids of squares, with second direction essentially constant, for example:
$$ \begin{tikzcd}[column sep=huge,row sep=huge]
A 
    \arrow[d, "f_1"] 
    \arrow[r,-,"\sim"] 
& {A'} 
    \arrow[d,"g_1"] 
    \arrow[r,-,"\sim"] 
& {A''}  
    \arrow[d,"h_1"]
    \arrow[r,-,"\sim"] 
& {A'''}  
    \arrow[d,"k_1"]    \\ 
B 
    \arrow[d, "f_2"] 
    \arrow[r,-,"\sim"]
    \arrow[shorten <=10pt,shorten >=10pt,ru,Rightarrow]
& {B'} 
    \arrow[d,"g_2"] 
    \arrow[r,-,"\sim"]
    \arrow[shorten <=10pt,shorten >=10pt,ru,Rightarrow]
& {B''}  
    \arrow[d,"h_2"]
    \arrow[r,-,"\sim"]
    \arrow[shorten <=10pt,shorten >=10pt,ru,Rightarrow]
& {B'''}  
    \arrow[d,"k_2"]    \\ 
C 
    \arrow[r,-,"\sim"]
    \arrow[shorten <=10pt,shorten >=10pt,ru,Rightarrow] 
& {C'}
    \arrow[r,-,"\sim"]
    \arrow[shorten <=10pt,shorten >=10pt,ru,Rightarrow] 
& {C''}
    \arrow[r,-,"\sim"]
    \arrow[shorten <=10pt,shorten >=10pt,ru,Rightarrow]
& {C'''}
\end{tikzcd} \quad = \quad $$
$$\begin{tikzcd}[column sep=huge,row sep=huge,labels=description]
\phantom{first line here} & \phantom{first line here} & \phantom{first line here} \\
{A}
    \arrow[bend left=80pt]{r}[name=1]{f_1}
    \arrow[bend left=20pt]{r}[name=2]{g_1}
    \arrow[bend right=20pt]{r}[name=3]{h_1}
    \arrow[bend right=80pt]{r}[name=4]{k_1}
& {B}
    \arrow[bend left=80pt]{r}[name=5]{f_2}
    \arrow[bend left=20pt]{r}[name=6]{g_2}
    \arrow[bend right=20pt]{r}[name=7]{h_2}
    \arrow[bend right=80pt]{r}[name=8]{k_2}
& {C}
    \arrow[shorten <=3pt,shorten >=3pt,Rightarrow,to path={(1) --  (2)}]{}
    \arrow[shorten <=3pt,shorten >=3pt,Rightarrow,to path={(2) --  (3)}]{}
    \arrow[shorten <=3pt,shorten >=3pt,Rightarrow,to path={(3) --  (4)}]{}
    \arrow[shorten <=3pt,shorten >=3pt,Rightarrow,to path={(5) --  (6)}]{}
    \arrow[shorten <=3pt,shorten >=3pt,Rightarrow,to path={(6) --  (7)}]{}
    \arrow[shorten <=3pt,shorten >=3pt,Rightarrow,to path={(7) --  (8)}]{} \\
\phantom{last     line} & \phantom{last     line} & \phantom{last     line}
\end{tikzcd} $$

One consequence of the globe condition is that any $k$-morphism has unique $i$-source and $i$-target (where $0 \leq i < k \leq n$). For example, when $n=2$, $X_{1,1}$ is the space of all $2$-morphisms in $X$. Face maps in $\Delta$ induce four boundary maps:

$$\begin{tikzcd}[column sep=huge,row sep=huge] 
\phantom{a}
& X_{0,1}
& \phantom{a} \\
X_{0,1}
& X_{1,1}
    \arrow[u,"\partial^{-}_1"]
    \arrow[l,"\partial^{+}_1"]
    \arrow[d,"\partial^{-}_2"]
    \arrow[r,"\partial^{+}_2"]
& X_{1,0} \\ 
\phantom{a}
& X_{1,0}
\end{tikzcd}$$

\begin{center}
\begin{tikzpicture}[column sep=huge,row sep=huge]
\node(1)   at (4,3)  {$(A \xrightarrow{a} A')$};
\node(2) at (0,0)  {$(B \xrightarrow{b} B')$};
\node(3) at (4,0)  {$\left( \begin{tikzcd}[column sep=huge,row sep=huge] 
A
    \arrow[d, "f"'] 
    \arrow[r,-,"\sim","a"'] 
& {A'} 
    \arrow[d,"g"] \\ 
B
    \arrow[r,-,"\sim","b"'] 
    \arrow[shorten <=10pt,shorten >=10pt,ru,Rightarrow] 
& {B'} 
\end{tikzcd}  \right)$};
\node(4) at (8,0) {$(A' \xrightarrow{g} B')$};
\node(5) at (4,-3)  {$(A \xrightarrow{f} B)$};
\draw[->,shorten >=10pt,shorten <=10pt] 
  (3) -- (1);
\draw[->,shorten >=10pt,shorten <=10pt] 
  (3) -- (2);
\draw[->,shorten >=10pt,shorten <=10pt] 
  (3) -- (4);
\draw[->,shorten >=10pt,shorten <=10pt] 
  (3) -- (5);
\end{tikzpicture}
\end{center}

However, we see that images of $\partial^{\pm}_1$ are always degenerate, as arrows in $X_{0,1}$ are essentially constant. The two non-degenerate boundaries gives $1$-source ($s_1:=\partial^{-}_2$) and $1$-target ($t_1:=\partial^{+}_2$) respectively; while we have  $0$-source $s_0:=\mathrm{dom} \circ \partial^{-}_1 = \mathrm{dom} \circ \partial^{-}_2$ and $0$-target $t_0:=\mathrm{cod} \circ \partial^{+}_1 = \mathrm{cod} \circ \partial^{+}_1$.

In general, up to weak equivalence, we have \textbf{unique $i$-source and $i$-target maps} induced from composition series of face maps of various coordinates, for $0 \leq i \leq k \leq n$, $$ s^{(k)}_i,t^{(k)}_i:X({\underbrace{[1],\dots,[1]}_{k \; \text{copies}},\underbrace{[0],\dots,[0]}_{(n-k) \; \text{copies}}}) \to X({\underbrace{[1],\dots,[1]}_{i \; \text{copies}},\underbrace{[0],\dots,[0]}_{(n-i) \; \text{copies}}}) $$ satisfy the following conditions ($0 \leq i < j \leq k \leq n$)
$$ s^{(k)}_i=s^{(j)}_i \circ s^{(k)}_j = s^{(j)}_i \circ t^{(k)}_j $$
$$ t^{(k)}_i=t^{(j)}_i \circ s^{(k)}_j = t^{(j)}_i \circ t^{(k)}_j $$
$$ s^{(k)}_i=s^{(i+1)}_i \circ \cdots \circ s^{(k)}_{k-1} $$
$$ t^{(k)}_i=t^{(i+1)}_i \circ \cdots \circ t^{(k)}_{k-1} $$

Segal conditions on a $n$-fold complete Segal space $X$ defines \textbf{compositions of all degrees in all directions}, up to weak equivalence. In specific, when $n=2$, $X_{1,0} \times_{X_{0,0}} X_{1,0} \simeq X_{2,0} \to X_{1,0} $ gives compositions of $1$-morphisms as usual:

$$\left( \begin{tikzcd}
\phantom{a}  & \cdot \arrow[rd,"b",color=red] & \phantom{a} \\
\cdot \arrow[ru,"a",color=red] & \phantom{a} & \cdot
\end{tikzcd} \right) 
\mapsto
\left( \begin{tikzcd}
\phantom{a}  & \cdot \arrow[rd,"b",color=red] & \phantom{a} \\
\cdot \arrow[ru,"a",color=red] \arrow[rr,"a . b",color=blue] & \phantom{a} & \cdot
\end{tikzcd} \right) 
\mapsto 
\left( \begin{tikzcd}
\cdot \arrow[r,"a . b",color=blue] & \cdot
\end{tikzcd} \right)$$

Then $X_{1,1} \times_{X_{1,0}} X_{1,1} \simeq X_{1,2} \to X_{1,1} $ gives vertical compositions of $2$-morphisms, while $X_{1,1} \times_{X_{0,1}} X_{1,1} \simeq X_{2,1} \to X_{1,1} $ gives horizontal compositions of $2$-morphisms. Notice that globe condition implies that $X_{0,1} \simeq X_{0,0}$, so horizontal compositions can also be written as $X_{1,1} \times_{X_{0,0}} X_{1,1} \to X_{1,1}$.

Recall that in the theory of bi-categories (e.g. $\mathsf{Cat}$, the category of small categories with functors and natural transformations), we can compose a $2$-morphism with a $1$-morphism in the horizontal direction, vice versa. This can be decomposed into two steps: firstly, we use the degeneracy maps to send the $1$-morphism to a degenerate $2$-morphism, then we can compose these two $2$-morphisms horizontally.
$$ \begin{tikzcd}[row sep=huge]
{A} 
    \arrow[bend right]{r}[name=D,label=below:$g$]{}
    \arrow[bend left]{r}[name=U,label=above:$f$]{}
& {B}
    \arrow[r,"h"]
& {C}
    \arrow[shorten <=3pt,Rightarrow,from=U,to=D,"\alpha"]{}
\end{tikzcd} \rightsquigarrow 
\begin{tikzcd}[row sep=huge]
{A} 
    \arrow[bend right]{r}[name=D,label=below:$g$]{}
    \arrow[bend left]{r}[name=U,label=above:$f$]{}
& {B}
    \arrow[bend right]{r}[name=F,label=below:$h$]{}
    \arrow[bend left]{r}[name=I,label=above:$h$]{}
& {C}
    \arrow[shorten <=3pt,Rightarrow,from=U,to=D,"\alpha"]{}
    \arrow[shorten <=3pt,equal,from=I,to=F]{}
\end{tikzcd} \rightsquigarrow 
\begin{tikzcd}[column sep=huge,row sep=huge]
{A} 
    \arrow[bend right]{r}[name=D,label=below:$g.h$]{}
    \arrow[bend left]{r}[name=U,label=above:$f.h$]{}
& {C}
    \arrow[shorten <=3pt,Rightarrow,from=U,to=D,"\alpha.\mathrm{id}_h"]{}
\end{tikzcd}$$

In general, Segal conditions (plus degeneracy maps) give $i$-compositions of $(i+j)$-morphism with $(i+k)$-morphism(...)
\end{remark}


\section{Cubical sets with connection}

In this section, we would like to provide a description of the cubical model that we are going to model higher infinity categories in the later sections. This model first appears in algebraic topology, as a tool of generalize some classical theorems, such as Seifert–van Kampen theorem, Dold-Kan correspondence, to non-Abelian, higher dimensional situations. (See non-Abelian algebraic topology)

\begin{mydef}
The \textbf{geometric $n$-cube} $\square^n$ is defined to be the topological space $[0,1]^n$, where $[0,1]$ is the unit interval. Geometric \textbf{face maps} $D^{\pm}_i:\square^n \to \square^{n+1}$ ($1 \leq i \leq n+1$) are defined as $$D^+_i: (x_1,\dots,x_n) \mapsto (x_1,\dots,x_{i-1},1,x_{i},\dots,x_n)$$    $$D^-_i: (x_1,\dots,x_n) \mapsto (x_1,\dots,x_{i-1},0,x_{i},\dots,x_n)$$ and geometric \textbf{degeneracy maps} $S_i:\square^{n+1} \to \square^n$ ($1 \leq i \leq n+1$) are defined as: $$ S_i: (x_1,\dots,x_{n+1}) \mapsto (x_1,\dots,x_{i-1},x_{i+1},\dots,x_n)$$ In addtion, for integer $n \geq 1$, we define the geometric \textbf{connection maps} to be $G^{\pm}_i: \square^{n+1} \to \square^n$ ($1 \leq i \leq n$) $$G^+_i: (x_1,\dots,x_{n+1}) \mapsto (x_1, \dots,x_{i-1},\mathrm{max}(x_i,x_{i+1}),x_{i+2},\dots,x_{n+1})) $$    $$ G^-_i: (x_1,\dots,x_{n+1}) \mapsto (x_1, \dots,x_{i-1},\mathrm{min}(x_i,x_{i+1}),x_{i+2},\dots,x_{n+1})) $$ The \textbf{cube category} $\square$ is defined to be the subcategory of $\mathsf{Top}$ with geometric $n$-cubes as objects, spanned by geometric face and degeneracy maps. The \textbf{cube category with connection} $\square^c$ is the subcategory of $\mathsf{Top}$ with geometric $n$-cubes as objects, spanned by geometric face, degeneracy and connection maps. Any small category equivalent to $\square$ (resp. $\square^c$) should also be regarded as the cube category (resp. with connection); indeed, we can define these category in a pure combinatorial way (...)
\end{mydef}

\appendix

\section{Strict $n$-categories}

This appendix includes some basics on strict $n$-categories. The definition of strict $n$-category is naive but intuitive. The whole content can be built rigorously inside the framework of ordinary category theory, so no advanced tools of homotopy theory from algebraic topology are required. On the other hand, it shares common drawbacks with ordinary categories, i.e. we sometimes have to use \textbf{strict equality} of morphisms, which is something ill-defined \textit{from a category-theoretic point of view}. We would also find some objects are \textit{equivalent in apriori} but not equivalent in our model.

Let's recall some basics about enriched categories.

\begin{mydef}
Suppose $(V,\otimes,I)$ is a symmetric monoidal category. Then a category $M$ is realled \textbf{$V$-enriched} if there is a bifunctor (called \textbf{$V$-enriched hom}) $[\cdot,\cdot]:M^{op}\times M \to V$ such that 

\begin{enumerate}
    \item there is a unitor takes any object $A$ in $M$ to a morphism in $V$: $$ \mathbbm{1}_A: I \to [A,A] $$ 
    \item there is a compositor take any objects $A,B,C$ in $M$ to a morphism in $V$: $$ \mathrm{com}_{A,B,C}:[A,B] \otimes [B,C] \to [A,C]$$
    \item the following diagram commutes up to natural isomorphism $$ \begin{tikzcd}[column sep=huge,row sep=huge] {[A,B] \otimes [B,C] \otimes [C,D]} \arrow[d, "{\mathrm{id} \otimes\mathrm{com}_{B,C,D}}"'] \arrow[r, "{\mathrm{com}_{A,B,C}\otimes \mathrm{id}}"] & {[A,C] \otimes [C,D]} \arrow[d, "{\mathrm{com_{A,C,D}}}"] \\ {[A,B] \otimes [B,D]} \arrow[r, "{\mathrm{com}_{A,B,D}}"'] \arrow[shorten <=30pt,shorten >=30pt,ru,Leftrightarrow]& {[A,D]} \end{tikzcd}$$
    \item the following diagram commutes up to natural isomorphisms $$ \begin{tikzcd}[column sep=huge,row sep=huge] {I \otimes [A,B]} \arrow[d, "\mathbbm{1}_A \otimes \mathrm{id}"'] & {[A,B]} \arrow[d,equal] \arrow[l, "\cong"'] \arrow[r, "\cong"] \arrow[shorten <=30pt,shorten >=30pt,ld,Leftrightarrow] \arrow[shorten <=30pt,shorten >=30pt,rd,Leftrightarrow] & {[A,B] \otimes I} \arrow[d, "\mathrm{id} \otimes\mathbbm{1}_B"] \\ {[A,A]\otimes [A,B]} \arrow[r, "{\mathrm{com}_{A,A,B}}"']   & {[A,B]} & {[A,B] \otimes [B,B]} \arrow[l, "{\mathrm{com}_{A,B,B}}"] \end{tikzcd}$$
\end{enumerate} 
Above constructions are required to be natural in suitable ways.

\end{mydef}

\begin{remark}
Here we view a category as an algebraic system \textit{externally}, i.e. we compose arrows \textbf{from left to right} and use $f.g$ to denote $f$ followed by $g$. However, when we discuss compositions \textit{inside} a category, such as drawing a commutative diagram, we still adapt the convention of composing arrows \textbf{from right to left} and use $g \circ f$ to denote $f$ followed by $g$. Which convention is used should be clear according to the content.
\end{remark}

\begin{eg}
\begin{enumerate}
    \item Any locally small category is enriched over $\mathsf{Set}$, with Cartesian product as monoidal product, singleton as monoidal unit and the usual hom bifunctor as enriched hom.
    \item Any (symmetric) monoidal closed category is enriched over itself, with its internal hom as enriched hom; in specific, any Cartesian closed category is enriched over itself.
    \item Any additive category is enriched over $(\mathsf{Ab},\otimes,\mathbb{Z})$, i.e. the category of Abelian groups ($\mathbb{Z}$-modules) with usual tensor product as monoidal product, $\mathbb{Z} $ as monoidal unit. In particular, for any ring $R$, the category of left $R$-modules, $R\mathsf{Mod}$, is enriched over $(\mathsf{Ab},\otimes,I)$.
\end{enumerate}
\end{eg}

\begin{remark}
If we always assume $V$ and $M$ to be locally small, then there is a natutral isomorphism of bifunctors $$ \mathbf{Hom}_V(I,[\cdot,\cdot]) \cong \mathbf{Hom}_M(\cdot,\cdot)$$ which suggests where the name of enriched hom comes.
\end{remark}

\begin{mydef}
Suppose $M,N$ are two (small) $V$-enriched categories, then a \textbf{$V$-enriched functor} $F:M \to N$ has the following data:
\begin{enumerate}
    \item on the object level, $F: \mathrm{Ob}(M) \to \mathrm{Ob}(N)$
    \item on the morphism level, for any two object $A,B$ in $M$, there is a morphism in $V$ (natural in $A$ and $B$)$$ F_{A,B}:[A,B]_M \to [F(A),F(B)]_N $$
    \item $F$ commutes with unitor and compositor, up to natural isomorphisms: $$ \begin{tikzcd}[column sep=huge,row sep=huge] [A,B] \otimes [B,C] \arrow[d, "{F_{A,B} \otimes F_{B,C}}"'] \arrow[r, "{\mathrm{com}_{A,B,C}}"] & {[A,C]} \arrow[d, "F_{A,C}"] \\ {[FA,FB] \otimes [FB,FC]} \arrow[r, "{\mathrm{com}_{FA,FB,FC}}"'] \arrow[shorten <=30pt,shorten >=30pt,ru,Leftrightarrow]& {[FA,FC]} \end{tikzcd} $$    $$\begin{tikzcd}[column sep=huge,row sep=huge] I \arrow[r, "\mathbbm{1}_{A}"] \arrow[d,equal] & {[A,A]} \arrow[d, "{F_{A,A}}"] \\ I \arrow[r, "\mathbbm{1}_{FA}"'] \arrow[shorten <=10pt,shorten >=10pt,ru,Leftrightarrow] & {[FA,FA]} \end{tikzcd}$$
\end{enumerate}
The collection of all small $V$-enriched categories and $V$-enriched functors between them forms a category, which we denote as $V\mathsf{enrich}$.

\end{mydef}

\begin{mydef}
Let $\mathsf{Cat}_0:=\mathsf{Set}$ and for any positive integer $k$, we define $\mathsf{Cat}_k:=\mathsf{Cat}_{k-1}\mathsf{enrich}$, called the \textbf{category of small strict $n$-categories}, where $\mathsf{Cat}_{k-1}$ is viewed as a symmetric monoidal category via categorical product. It is immediate that $\mathsf{Cat}_{1}$ coincides with the category of small categories, $\mathsf{Cat}$ (viewed as ordinary category, i.e. forget natural transformations between functors).
\end{mydef}

\begin{mydef}

There are obvious embeddings of categories $ \iota_n: \mathsf{Cat}_{n-1} \hookrightarrow \mathsf{Cat}_n$ (where $n$ is a positive integer), which is inductively defined with $\iota_1: \mathsf{Set} \hookrightarrow \mathsf{Cat}$ be the embedding sending any set to the discrete groupoid generated by it. One can verify that $\iota_n$ has right adjoint, denoted as $\partial: \mathsf{Cat}_n \to \mathsf{Cat}_{n-1}$, which sends any strict $n$-category to the maximal strict $(n-1)$-subcategory inside it (or equivalently, deletes all $n$-morphisms). In convention, for any set $S$, we set $\partial S:= \varnothing$.

\end{mydef}

\begin{remark}
Here are some subtleties: recall that $E$ is the connected groupoid with two objects and a pair of non-trivial isomorphisms between them. Its boundary, $\partial E$, is a set with two elements. Also, let $I$ to be the delooping of $\mathbb{Z}$ (i.e. a category with a single object and one generator of non-trivial isomorphism), then $E$ and $I$ are equivalent as categories (in fact, $I$ is a retract of $E$). However, $\partial I$ is a singleton, so $\partial E$ and $\partial I$ are not equivalnet as sets. In general, two equivalent strict $n$-categories (as equivalent enriched categories, i.e. they have isomorphic retracts) will have non-equivalent strict $(n-1)$-categories as boundaries. It is clear that these subtleties come from non-trivial higher isomorphisms between distinct objects. One way of going around these subtleties is, instead of taking boundaries naively, we first delete all \textit{non-invertible} $n$-morphisms from a strict $n$-category, take its skeleton and then forget all invertiable $n$-morphisms left. This hint us that instead of considering all strict $n$-categories, it makes more sense to consider strict $n$-categories without non-trivial higher isomorphisms between distinct objects. This leads us to the following definition.
\end{remark}

\begin{mydef}
A strict $n$-category is called \textbf{gaunt} if there are no non-trivial $k$-isomorphisms between distinct objects for $1 \leq k \leq n$. The full subcategory spanned by gaunt (strict) $n$-categories is denoted as $\mathsf{Gaunt}_n$. Here are some basic observations about gaunt $n$-categories:
\begin{enumerate}
    \item the image of $\mathsf{Gaunt}_{n-1}$ under $\iota_n$ lies in $\mathsf{Gaunt}_n$.
    \item any $n$-subcategory of a gaunt $n$-category is gaunt.
    \item boundary of a gaunt $n$-category is gaunt.
    \item the inclusion $\mathsf{Gaunt}_n \hookrightarrow \mathsf{Cat}_n$ admits a left adjoint (localization) $L_n:\mathsf{Cat}_n \to \mathsf{Gaunt}_n$ and $L_n$ followed by the inclusion is idempotent.
\end{enumerate}

\end{mydef}

\begin{mydef}
Given any strict $n$-category $C$, we construct a strict $(n+1)$-category, which is called the \textbf{suspension} of $C$ and denoted as $\sigma C$, defined as follows: objects of $C$ are two distinct elements, $\bot,\top$; on the level of morphisms, $\sigma C(\bot,\top):=C$, $\sigma C(\bot,\bot) \cong * \cong \sigma C(\top,\top)$ and $\sigma C(\top,\bot) \cong \varnothing$. Suspension can be extend to a functor $\sigma: \mathsf{Cat}_n \to \mathsf{Cat}_{n+1}$: for any functor of strict $n$-categories $F:C \to D$, we define $\sigma F:\sigma C \to \sigma D$ by setting $\sigma F(\bot):=\bot, \sigma F(\top):=\top$, $$ \sigma F_{\bot,\top}:=F:C \to D $$ $$ \sigma F_{\bot,\bot}:=\mathrm{id}: * \to * $$ $$ \sigma F_{\top,\top}:=\mathrm{id}:* \to * $$ $$ \sigma F_{\top,\bot}:=\mathrm{id}: \varnothing \to \varnothing $$ One can verify that this indeed gives a functor.

We define \textbf{$n$-globes} as strict $n$-categories inductively: $G_0:=*$ i.e. $0$-globe is singleton; for positive integer $n$, $G_n:=\sigma G_{n-1}$. Clearly, $0$-globe is gaunt, and suspension of any gaunt category is still gaunt; so $n$-globe is gaunt for any $n \in \mathbb{N}$. Using iterated embedding $\iota$, we can view $G_k$ as strict $n$-category, provided $k \leq n$. The full subcategory spanned by $G_k$ (for $0 \leq k \leq n$) in $\mathsf{Cat}_n$ is denoted as $\mathbb{G}_n$, the $n$-globe category.
\end{mydef}

\begin{remark}
For positive integer $n$, $$\partial G_n \cong G_{n-1} \sqcup_{\partial G_{n-1}} G_{n-1}$$ where $\partial G_{n-1} \hookrightarrow G_{n-1}$ is the essentially unique embedding. Recall that we set $\partial G_0 = \varnothing$, so $\partial G_1 \cong * \sqcup *$. We can compare this with a similar topological result: let $D^n$ be $n$-dimensional closed disc, $S^n \cong \partial D^n$ be $n$-dimensional sphere, then $$ S^n \cong D^{n-1} \sqcup_{S^{n-1}} D^{n-1} $$ where $S^{n-1} \hookrightarrow D^{n-1}$ is the embedding of $S^{n-1}$ to the boundary of $D^{n-1}$.
\end{remark}

\begin{remark}
Suppose $i$ is a non-negative integer and $j,k$ are positive integers. Then we have the following diagram to be a pushout diagram $$ \begin{tikzcd} G_{i+j} \times_{G_i} G_{i+k} & G_{i+j} \sqcup_{G_i} G_{i+k} \arrow[l] \\ G_{i+k} \sqcup_{G_i} G_{i+j} \arrow[u] & \sigma^{i+1}(G_{j-1} \times G_{k-1})  \arrow[l] \arrow[u] \end{tikzcd}$$ where $G_{i+j} \times_{G_i} G_{i+k} $ is the limit of the the cospan $G_{i+j}\to G_{i} \leftarrow G_{i+k}$ (non-degenerate projections), and $G_{i+j} \sqcup_{G_i} G_{i+k} $ is the colimit of the span $G_{i+j} \hookleftarrow G_i \hookrightarrow G_{i+k} $ where the left arrow embeds $G_i$ into the \textbf{terminal} $i$-morphism in $G_{i+j}$ and the right arrow embeds $G_i$ into the \textbf{initial} $i$-morphism in $G_{i+k}$. In contrast, $G_{i+k} \sqcup_{G_i} G_{i+j}$ is the colimit of the span $G_{i+k} \hookleftarrow G_i \hookrightarrow G_{i+j}$, where the left arrow embeds $G_i$ into the \textbf{terminal} $i$-morphism in $G_{i+k}$ and the right arrow embeds $G_i$ into the \textbf{initial} $i$-morphism in $G_{i+j}$.

These complicated diagrams actually say in words: higher commutative diagrams can be constructed from elementary blocks - globes and operations such as suspensions, products and pushouts. For the simplest case, let $i=0,j=k=1$, then $G_1 \times_{G_0} G_1$ is just a commutative square, i.e. two pairs of composable $1$-morphisms with equal compositions. Then this diagram clearly commutes: $$\begin{tikzcd} G_{1} \times_{G_0} G_{1} & G_{1} \sqcup_{G_0} G_{1} \arrow[l] \\ G_{1} \sqcup_{G_0} G_{1} \arrow[u] & G_1 \cong \sigma(G_{0} \times G_{0})  \arrow[l] \arrow[u] \end{tikzcd}$$

Now let's move on to the case when $i=1,j=k=1$. $G_2 \times_{G_1} G_2$ should be viewed as a strict $2$-commutative square for vertical composition (i.e. the boundary of a cube with a pair of opposite faces constant).
$$ \begin{tikzcd}[column sep=huge,row sep=huge] 
\phantom{a} 
& \phantom{a} 
    \arrow[Rightarrow,rdd,shorten <=10pt,shorten >=45pt]   
    \arrow[Rightarrow,ld,dashed,shorten <=15pt,shorten >=15pt] 
& \phantom{a} \\ 
{\bot} 
    \arrow[rr, bend left,dashed] 
    \arrow[rr, bend left=75] 
    \arrow[rr, bend right] 
    \arrow[rr, bend right=75] 
& 
& {\top} 
    \arrow[Rightarrow,ld,shorten <=15pt,shorten >=15pt]    \\ 
\phantom{a}  
& \phantom{a} 
    \arrow[Leftarrow,luu,dashed,shorten <=10pt,shorten >=45pt]  
& \phantom{a} 
\end{tikzcd}  $$

For $i=0,j=k=2$, $G_2 \times_{G_0} G_2$ is a strict $2$-commutative square for horizontal composition.

$$ \begin{tikzcd}[column sep=huge,row sep=huge]
{\bot \bot} 
    \arrow[bend right]{d}[name=A]{}
    \arrow[bend left]{d}[name=B]{}
    \arrow[bend right]{r}[name=D]{}
    \arrow[bend left]{r}[name=C]{}
& {\top \bot} 
    \arrow[bend right]{d}[name=E]{}
    \arrow[bend left]{d}[name=F]{} \\
{\bot \top} 
    \arrow[bend right]{r}[name=H]{}
    \arrow[bend left]{r}[name=G]{}
& {\top \top}
    \arrow[shorten >=3pt,Rightarrow,to path={(A) -- (B)}]{}
    \arrow[shorten <=3pt,Rightarrow,to path={(C) -- (D)}]{}
    \arrow[shorten >=3pt,Rightarrow,to path={(E) -- (F)}]{}
    \arrow[shorten <=3pt,Rightarrow,to path={(G) -- (H)}]{}
\end{tikzcd} $$

It is commutative in the sense that, the four intermediate states of $1$-morphisms and the $2$-arrows between them form a strict $2$-commutative square (of vertical compositions):

$$\begin{tikzpicture}[column sep=huge,row sep=huge]
%\node(1){\phantom{a}};
\node(A) at (2,2) {
    \begin{tikzcd}
    \cdot
        \arrow[r,bend left]
    & \cdot
        \arrow[r,bend left]
    & \cdot
    \end{tikzcd}};
\node(B)  (0,0) {\begin{tikzcd}
    \cdot
        \arrow[r,bend left]
    & \cdot
        \arrow[r,bend right]
    & \cdot
    \end{tikzcd}};
%\node(4)[right of=3]{\phantom{$\begin{matrix} 1 & 1 & 1 \\ 1 & 1 & 1 \end{matrix}$}};
\node(C) at (4,0) {\begin{tikzcd}
    \cdot
        \arrow[r,bend right]
    & \cdot
        \arrow[r,bend left]
    & \cdot
    \end{tikzcd}};
\node(D) at (2,-2) {\begin{tikzcd}
    \cdot
        \arrow[r,bend right]
    & \cdot
        \arrow[r,bend right]
    & \cdot
    \end{tikzcd}};
\draw[double,double equal sign distance,-implies,shorten >=35pt,shorten <=10pt] 
  (A) -- (B);
\draw[double,double equal sign distance,-implies,shorten >=10pt,shorten <=5pt] 
  (A) -- (C);
\draw[double,double equal sign distance,-implies,shorten >=10pt,shorten <=5pt] 
  (C) -- (D);
\draw[double,double equal sign distance,-implies,shorten >=15pt,shorten <=30pt] 
  (B) -- (D);
\end{tikzpicture}$$

In general, $G_{i+j} \times_{G_{i}} G_{i+k}$ gives a strict higher commutative diagram for composition of $i$-correspondences (which we would define later in this section) between a $(i+j)$-globe and a $(i+k)$-globe.

\end{remark}

Before going into the general definition of the category of $k$-correspondences between gaunt $n$-categories (note that we choose to consider \textbf{gaunt} categories only due to technical reasons given before), let's review some familiar examples as special cases.

\begin{eg}
The \textbf{Morita category of rings} is defined as follows: 
\begin{enumerate}
    \item its objects are associative rings with multiplicative identities, $R,S,T,\dots$
    \item $1$-morphisms are bimodules, and we use $\tensor[_R]{M}{_S}$ to denote a left-$R$ right-$S$ module $M$. Compositions of $\tensor[_R]{M}{_S}$ and $\tensor[_S]{M}{_T}$ are given (non-strictly) by tensor product ${}_R M \otimes_S N_T$
    \item $2$-morphisms are bimodule homomorphisms, and vertical compositions are just compositions of bimodule homomorphisms; horizontal compositions are induced by tensor products of bimodules. Thus, compositions of $1$-morphisms are defined uniquely up to a $2$-isomorphism.
\end{enumerate} 

Recall that a left-$R$ right-$S$ module $M$ is the same as a bifunctor $M:S^{op} \times R \to \mathsf{Ab}$ (formally speaking, here $R$ and $S$ are \textbf{deloopings} of the corresponding rings), and a bimodule homomorphism is just a natural transformation between two such bifunctors.
\end{eg}

This motivates us to generalize these constructions to all categories.

\begin{mydef}
The category of $1$-correspondences (so called \textbf{profunctors}) between (ordinary) small categories, denoted as $\mathsf{Corr}_1^1$, is defined as:
\begin{enumerate}
    \item its objects are small categories
    \item $1$-morphisms are bifunctors $F:A^{op}\times B \to \mathsf{Set}$
    \item $2$-morphisms are natural transformations between bifunctors and vertical compositions are given by usual vertical compostions of natural transformations.
    \item compostions of $1$-morphisms and horizontal compositions of $2$-morphisms induced by them come from the coend construction: suppose we now have two $1$-morphisms $F:A^{op}\times B \to \mathsf{Set}$ and $G:B^{op}\times C \to \mathsf{Set}$, then $$ F.G(-,-) := \int^{b:B}{F(-,b) \times G(b,-)}:A^{op}\times C \to \mathsf{Set} $$
\end{enumerate}
Recall that our convention for external compostions of morphisms is from left to right.
\end{mydef}

These constructions can be generalized further to any higher gaunt categories.

\begin{mydef}
We define the \textbf{category of $k$-correspondences between gaunt $n$-categories}, $\mathsf{Corr}^k_n$ , (provided $k \leq n$) inductively as following: $\mathsf{Corr}^0_n:=\mathsf{Gaunt}_n$ and then for $\mathsf{Corr}^k_n$ with $k \geq 1$,
\begin{enumerate}
    \item its objects are gaunt $n$-categories  $A,B,C,\dots$
    \item its $1$-morphisms are bifunctors $F:A^{op}\times B \to \mathsf{Corr}^{k-1}_{n-1}, G:B^{op} \times C \to \mathsf{Corr}^{k-1}_{n-1}$, and their composition is given by the coend $$ F.G(-,-):= \int^{b:B}{F(-,b)\times G(b,-)}:A^{op} \times C \to \mathsf{Corr}^{k-1}_{n-1}$$
    \item information of higher morphisms is contained in the enriched hom space $[A^{op} \times B,\mathsf{Corr}^{k-1}_{n-1}]$, as a gaunt $(n-1)$-category.
\end{enumerate}
\end{mydef}

\begin{remark}
To show that this iterated definition of $\mathsf{Corr}^k_n$ is well-behaved, one should verify that, for $l<k,m<n$, $\mathsf{Corr}^l_m$ is a cocomplete, locally presentable gaunt $(m+1)$-category (notice that $\mathsf{Corr}^l_m$ cannot be small $(m+1)$-category, so strictly speaking this is not a hom space \textit{inside} $\mathsf{Gaunt}_{m+1}$; nevertheless, one can still talk about \textit{large} gaunt $n$-categories, but we omit the details here for briefness).
\end{remark}

We below give an equivalent characterization of the category of $k$ correspondences.

\begin{prop}
$$\mathsf{Corr}^k_n \simeq (\mathsf{Gaunt}_n)_{/G_k}$$ where the right hand side denotes the sliced category of $\mathsf{Gaunt}_n$ over $k$-globe $G_k$.
\end{prop}

\begin{proof}
We can prove this claim inductively. For $k=0$, the claim $\mathsf{Corr}^0_n \simeq \mathsf{Gaunt}_n \simeq (\mathsf{Gaunt}_n)_{/G_k}$ obviously holds. Assume $$\Phi^{k-1}_{n-1}:\mathsf{Corr}^{k-1}_n \simeq (\mathsf{Gaunt}_{n-1})_{/G_{k-1}}$$ and we construct a functor $\Phi^k_n:\mathsf{Corr}^k_n \to (\mathsf{Gaunt_n})_{/G_k}$ as follows: suppose $F:A^{op} \times B \to \mathsf{Corr}^{k-1}_{n-1}$ is a $k$-correspondence between gaunt $n$-categories, then let $X$ be the gaunt $n$-category with objects: $\mathrm{Ob}(X):=\mathrm{Ob}(A) \sqcup \mathrm{Ob}(B)$ and enriched hom spaces: for objects $a,a'$ in $A$, $b,b'$ in $B$, $$ X(a,a'):=A(a,a') $$    $$X(b,b'):= B(b,b')$$    $$X(a,b):=\mathrm{dom}(\Phi^{k-1}_{n-1} \circ F(a,b))$$    $$X(b,a):=\varnothing$$ with compositions defined \textit{externally} $$ \mathrm{com}_{a',a,b}: X(a',a) \times X(a,b) \to X(a',b) $$    $$ (\phi,f) \mapsto \phi^*(f) $$    $$ \mathrm{com}_{a,b,b'}: X(a,b) \times X(b,b') \to X(a,b') $$    $$ (f,\psi) \mapsto \psi_*(f) $$ where $\phi^*=F(\phi,b):F(a,b) \mapsto F(a',b)$ and $\psi_*=F(a,\psi):F(a,b) \to F(a,b')$. Compositions of other morphisms are given naturally by compositions in $A,B$ respectively, and the fact that $\varnothing \times C \cong \varnothing \cong C \times \varnothing$ for any gaunt $n$-category $C$.

There is a canonical $n$-functor $p:X \to G_k$ such that $ p^{-1}(\bot)=\mathrm{Ob}(A)$ and $ p^{-1}(\top)=\mathrm{Ob}(B) $. On the level of morphisms, $p(a,a'): X(a,a') \to G_k(\bot,\bot) \cong *$ and $p(b,b'):X(b,b') \to G_k(\top,\top) \cong *$ are the unique final morphisms, and $p(b,a):X(b,a) \cong \varnothing \to \varnothing \cong G_k(\top,\bot)$ all equal the identity on the initial object. Finally, $p(a,b):X(a,b) \to G_k(\bot,\top) \cong G_{k-1}$ can be defined to be equal to $\Phi^{k-1}_{n-1} \circ F(a,b) \in (\mathsf{Gaunt}_{n-1})_{/G_{k-1}}$.

One can check that $\Phi^k_n: (F:A^{op} \times B \to \mathsf{Corr}^{k-1}_{n-1}) \mapsto (p:X \to G_k)$ found above is indeed a functor, and it is fully faithful and essentially surjective. (Reader could also reverse the above process to get a functor $(\mathsf{Gaunt}_{n-1})_{/G_{k-1}} \to \mathsf{Corr}^k_n $ and then show that these functors form a pair of equivalences directly.) Thus we indeed see $\mathsf{Corr}^k_n \simeq (\mathsf{Gaunt}_n)_{/G_k}$.
\end{proof}

Then we can prove some properties of $\mathsf{Corr}^k_n$ via those of $(\mathsf{Gaunt}_n)_{/G_k}$.

\begin{prop}
The category of $k$-correspondences between $n$-categories, modelled by $(\mathsf{Gaunt}_n)_{/G_k}$, is Cartesian closed. 
\end{prop}

\begin{proof}
The category of (small) gaunt $n$-categories is cocomplete and has all pullbacks, since $\mathsf{Gaunt}_n$ is a reflective subcategory (i.e. a localization) of $\mathsf{Cat}_n$, which is complete and cocomplete. In particular, cartesian products in $(\mathsf{Gaunt}_n)_{/G_k}$ are just pullbacks of cospans in $\mathsf{Gaunt}_n$ over $G_k$. 

For any $p:B \to G_k$, the induced functor $- \times_{G_k} B:(\mathsf{Gaunt}_n)_{/G_k} \to (\mathsf{Gaunt}_n)_{/G_k}$ preserves all colimits, for it can be factorized into $$ (\mathsf{Gaunt}_n)_{/G_k} \xrightarrow{p^*} (\mathsf{Gaunt}_n)_{/B} \xrightarrow{p_*} (\mathsf{Gaunt}_n)_{/G_k}$$ i.e. for any $q:A \to G_k$, $$ \left( \begin{tikzcd} A \arrow[d,color=red,"q"] \\ G_k \end{tikzcd} \right) \mapsto \left( \begin{tikzcd} A \times_{G_k} B \arrow[d] \arrow[r,color=red] &  B \arrow[d,"p"] \\ A \arrow[r,"q"] & G_k \end{tikzcd} \right) \mapsto \left( \begin{tikzcd} A \times_{G_k} B \arrow[d,color=red] \\ G_k \end{tikzcd} \right)$$ and the two factor functors actually form an adjoint pair $(p_* \dashv p^*)$

(...)
\end{proof}

\end{document}
